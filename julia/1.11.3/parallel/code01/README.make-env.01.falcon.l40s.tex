# This is to make the julia virtual environment to run the parallel code.

# Start julia interpreter.
julia

# Go into package mode.
]

# Create a new "project" (i.e., environment) by
# giving the full path and the last "name" is the name
# of the project and the name of the directory (same thing).
activate /home/ckuhlman/env-julia/falcon/benchmarking/BenchmarkTools

# Now add as many packages as you want.
add BenchmarkTools
add CUDA

# In general, keep adding.
# ...

# When done adding packages, get out of the "project."
# To get out of project, type "activate" again.
# Yes, activate, not deactivate.
# (Take it up with Julia.)
activate

# You are still in package mode. But no longer in the project.

# When with package mode, hit cntrl-C at the "pkg" prompt to end "package mode."
cntrl-c

<< You are now done.  You do not do anything below here; the material below
is just for reference.  To get out of the julia interpreter, type exit() >>


# Now you are back at the julia command line interpreter.

# Aside:

# 1. Look under /home/<user-name>/env-julia/falcon/benchmarking/BenchmarkTools
#     - to see the new project (environment).
#     - based on the path I gave above.
# 2. Look under /home/<username>/.julia/packages 
#     - to see the new package installed, for use with other projects.

# On the command line, ***within julia interpreter*** (i.e., type julia), to load an env
# and import julia packages, do these steps:
# 1. using Pkg

# Below, the project (env) name is BenchmarkTools.  Notice the "/." after it.
# 2. Pkg.activate("/home/ckuhlman/env-julia/falcon/benchmarking/BenchmarkTools/.")
# Now you can import, one a time, the packages in the project (env) that you
# want to use.

# 3. import BenchmarkTools

<< Do NOT do this now.  It is just showing how to add more packages to an existing project/virtual env. >>

# To add more packages (later) to an existing project (env):
1. julia
2. ]
3. activate /home/ckuhlman/env-julia/falcon/benchmarking/BenchmarkTools
4. # To see the packages in a project (env):
5. status
6. add CUDA
7. status
8. I get:
9. Status `~/env-julia/falcon/benchmarking/BenchmarkTools/Project.toml`
  [6e4b80f9] BenchmarkTools v1.6.0
  [052768ef] CUDA v5.7.3
10 # Get out of the env (project).
11. activate
12. # Get out of package.
13. cntrl-C
14. # Get out of Julia interpreter.
15. exit()










I got this:

        Info Packages marked with → are not downloaded, use `instantiate` to download
        Info Packages marked with ⌅ have new versions available but compatibility constraints restrict them from upgrading. To see why use `status --outdated -m`
  Downloaded artifact: CUDA_Runtime
Precompiling project...
  67 dependencies successfully precompiled in 79 seconds. 40 already precompiled.

(BenchmarkTools) pkg> instantiate

